\let\negmedspace\undefined
\let\negthickspace\undefined
\documentclass[journal]{IEEEtran}
\usepackage[a5paper, margin=10mm, onecolumn]{geometry}
%\usepackage{lmodern} 
\usepackage{tfrupee} 

\setlength{\headheight}{1cm} 
\setlength{\headsep}{0mm}     

\usepackage{gvv-book}
\usepackage{gvv}
\usepackage{cite}
\usepackage{amsmath,amssymb,amsfonts,amsthm}
\usepackage{algorithmic}
\usepackage{graphicx}
\usepackage{textcomp}
\usepackage{xcolor}
\usepackage{txfonts}
\usepackage{listings}
\usepackage{enumitem}
\usepackage{mathtools}
\usepackage{gensymb}
\usepackage{comment}
\usepackage[breaklinks=true]{hyperref}
\usepackage{tkz-euclide} 
\usepackage{listings}                                        
\def\inputGnumericTable{}                                 
\usepackage[latin1]{inputenc}                                
\usepackage{color}                                            
\usepackage{array}                                            
\usepackage{longtable}                                       
\usepackage{calc}                                             
\usepackage{multirow}                                         
\usepackage{hhline}                                           
\usepackage{ifthen}                                           
\usepackage{lscape}

\begin{document}

\bibliographystyle{IEEEtran}
\vspace{3cm}

\title{12.497}
\author{AI25BTECH11003 - Bhavesh Gaikwad}
{\let\newpage\relax\maketitle}

\renewcommand{\thefigure}{\theenumi}
\renewcommand{\thetable}{\theenumi}
\setlength{\intextsep}{10pt} 

\numberwithin{equation}{enumi}
\numberwithin{figure}{enumi}
\renewcommand{\thetable}{\theenumi}


\textbf{Question}: If the rank of a (5$\times$6) matrix $\vec{Q}$ is 4, then which one of the following statements is correct?

\hfill{(EE 2008)}

\begin{itemize}
    \item[a)]$\vec{Q}$ will have four linearly independent rows and four linearly independent columns.
    \item[b)]$\vec{Q}$ will have four linearly independent rows and five linearly independent columns.
    \item[c)]$\vec{Q}\vec{Q}^\top$ will be invertible.
    \item[d)]$\vec{Q}^\top\vec{Q}$ will be invertible 
\end{itemize}

\bigskip
 
\textbf{Solution:}\\

\textbf{Primary Analysis:}\\
Since rank($\vec{Q}$)=4 $\Rightarrow \; \therefore \vec{Q}$ will have four linearly independent rows and four linearly independent columns.\\

Option-A:\\
Correct Option by Primary Analysis itself.\\

Option-B:\\
Incorrect Option by Primary Analysis itself.\\

Option-C:\\
$\vec{Q}\vec{Q}^\top$ is a 5$\times$5 matrix.\\
Since, rank($\vec{Q}\vec{Q}^\top$) = rank($\vec{Q}$).\\
$\therefore$ rank($\vec{Q}\vec{Q}^\top$) = 4.\\
Since rank($\vec{Q}\vec{Q}^\top$)=4$<$5, Thus the 5$\times$5 matrix $\vec{Q}\vec{Q}^\top$ is singular $\left(|\vec{Q}\vec{Q}^\top|=0\right)$, hence not invertible.
\bigskip
Incorrect Option.\\

Option-D:\\
$\vec{Q}^\top\vec{Q}$ is a 6$\times$6 matrix.\\
Since, rank($\vec{Q}^\top\vec{Q}$) = rank($\vec{Q}$).\\
$\therefore$ rank($\vec{Q}^\top\vec{Q}$) = 4.\\
Since rank($\vec{Q}^\top\vec{Q}$)=4$<$6, Thus the 6$\times$6 matrix $\vec{Q}^\top\vec{Q}$ is singular $\left(|\vec{Q}^\top\vec{Q}|=0\right)$, hence not invertible.
\bigskip
Incorrect Option.\\\\

Thus, Only Option-A is correct.








\newpage
\textbf{Proof by Example}

Consider the $5\times6$ matrix $\vec{Q}$ of rank 4:
\begin{equation}
\vec{Q}
=
\myvec{
1 & 0 & 0 & 0 & 2 & 3\\
0 & 1 & 0 & 0 & 4 & 5\\
0 & 0 & 1 & 0 & 6 & 7\\
0 & 0 & 0 & 1 & 8 & 9\\
0 & 0 & 0 & 0 & 0 & 0
}.
\end{equation}

Option (a): Four independent rows and columns\\

Clearly rows of $\vec{Q}$ are linearly independent. Thus row rank of $\vec{Q}$ = 4\\
Clearly columns of $\vec{Q}$ are linearly independent. Thus column rank of $\vec{Q}$ =4\\
Thus (a) holds.\\\\

Option (b): Four independent rows and Five independent columns\\

Column rank cannot exceed 4.\\
Hence (b) is false.\\\\

Option (c): Invertibility of $\vec{Q}\,\vec{Q}^\top$\\

\begin{equation}
\vec{Q}\,\vec{Q}^\top
=
\myvec{
1&0&0&0&2&3\\
0&1&0&0&4&5\\
0&0&1&0&6&7\\
0&0&0&1&8&9\\
0&0&0&0&0&0
}
\,
\myvec{
1&0&0&0&0\\
0&1&0&0&0\\
0&0&1&0&0\\
0&0&0&1&0\\
2&4&6&8&0\\
3&5&7&9&0
}
=
\myvec{
14&23&33&43&0\\
23&42&59&77&0\\
33&59&86&111&0\\
43&77&111&146&0\\
0&0&0&0&0
}
\end{equation}
Since the 5th row (and column) is zero, $|\vec{Q}\,\vec{Q}^\top|=0$. Not invertible.\\ 
(c) is false.

\newpage

Option (d): Invertibility of $\vec{Q}^\top\vec{Q}$\\

\begin{equation}
\vec{Q}^\top\vec{Q}
=
\myvec{
1&0&0&0&0\\
0&1&0&0&0\\
0&0&1&0&0\\
0&0&0&1&0\\
2&4&6&8&0\\
3&5&7&9&0
}
\,
\myvec{
1&0&0&0&2&3\\
0&1&0&0&4&5\\
0&0&1&0&6&7\\
0&0&0&1&8&9\\
0&0&0&0&0&0
}
=
\myvec{
1&0&0&0&2&3\\
0&1&0&0&4&5\\
0&0&1&0&6&7\\
0&0&0&1&8&9\\
2&4&6&8&120&154\\
3&5&7&9&154&197
}
\end{equation}

Since, $|\vec{Q}^\top\vec{Q}|=0$. Not invertible.\\ 
(d) is false.

\bigskip
\noindent Therefore, only option (a) is valid.

\end{document}  
