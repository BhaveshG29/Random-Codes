\documentclass{beamer}
\usepackage[utf8]{inputenc}

\usetheme{Madrid}
\usecolortheme{default}
\usepackage{amsmath,amssymb,amsfonts,amsthm}
\usepackage{txfonts}
\usepackage{tkz-euclide}
\usepackage{listings}
\usepackage{adjustbox}
\usepackage{array}
\usepackage{tabularx}
\usepackage{gvv}
\usepackage{lmodern}
\usepackage{circuitikz}
\usepackage{tikz}
\usepackage{graphicx}

\setbeamertemplate{page number in head/foot}[totalframenumber]

\usepackage{tcolorbox}
\tcbuselibrary{minted,breakable,xparse,skins}



\definecolor{bg}{gray}{0.95}
\DeclareTCBListing{mintedbox}{O{}m!O{}}{%
  breakable=true,
  listing engine=minted,
  listing only,
  minted language=#2,
  minted style=default,
  minted options={%
    linenos,
    gobble=0,
    breaklines=true,
    breakafter=,,
    fontsize=\small,
    numbersep=8pt,
    #1},
  boxsep=0pt,
  left skip=0pt,
  right skip=0pt,
  left=25pt,
  right=0pt,
  top=3pt,
  bottom=3pt,
  arc=5pt,
  leftrule=0pt,
  rightrule=0pt,
  bottomrule=2pt,
  toprule=2pt,
  colback=bg,
  colframe=orange!70,
  enhanced,
  overlay={%
    \begin{tcbclipinterior}
    \fill[orange!20!white] (frame.south west) rectangle ([xshift=20pt]frame.north west);
    \end{tcbclipinterior}},
  #3,
}
\lstset{
    language=C,
    basicstyle=\ttfamily\small,
    keywordstyle=\color{blue},
    stringstyle=\color{orange},
    commentstyle=\color{green!60!black},
    numbers=left,
    numberstyle=\tiny\color{gray},
    breaklines=true,
    showstringspaces=false,
}
%------------------------------------------------------------

\title
{12.289}
\date{September 28, 2025}
\author 
{AI25BTECH11003 - Bhavesh Gaikwad}



\begin{document}


\frame{\titlepage}
\begin{frame}{Question}
The approximate eigenvalue of the matrix 
$$\vec{A} = \myvec{15 & 4 & 3 \\ 10 & 12 & 6 \\ 20 & 4 & 2}$$
obtained after two iterations of Power method, with the initial vector $\myvec{1 & 1 & 1}^\top$ is 

\hfill{(MA 2012)}

(a) 7.768 $\qquad$ (b)9.468 $\qquad$ (c)10.548 $\qquad$ (d)19.468
\end{frame}


\begin{frame}[fragile]
    \frametitle{Theoretical Solution}
Given: 
$$\vec{A} = \myvec{15 & 4 & 3 \\ 10 & 12 & 6 \\ 20 & 4 & 2}$$
$$\text{Let the initial vector be} \; \vec{x}_0 = \myvec{1 \\ 1 \\ 1}$$

\begin{equation}
\vec{x}_{1} = \vec{A}\vec{x}_0 = \myvec{22 \\ 28 \\ 26}    
\end{equation}


Let Normalization vector: $\vec{y}_i = \dfrac{1}{x_{max}}\vec{x}_i$\\
where, $x_{max}$ is the largest element in $\vec{x}_i$
\end{frame}



\begin{frame}[fragile]
    \frametitle{Theoretical Solution}
\begin{equation}
\vec{y}_1 = \frac{1}{28}\vec{x}_1   
\end{equation}

\begin{equation}
\vec{x}_2 = \vec{A}\vec{y}_1 = \frac{1}{7}\myvec{260 \\ 356 \\ 302} 
\end{equation}

\begin{equation}
    \vec{y}_2 = \frac{1}{356} \vec{x}_2 = \frac{1}{1246}\myvec{130 \\ 178 \\ 151}
\end{equation}
\end{frame}



\begin{frame}[fragile]
    \frametitle{Theoretical Solution}
Let $\lambda$ be the Dominant eigenvalue.\\

By Rayleigh-Quotient,
\begin{equation}
    \lambda = \dfrac{\vec{y}_2^\top\vec{A}\vec{y}_2}{\vec{y}_2^\top\vec{y}_2}
\end{equation}

\begin{align*}
    \boxed{\therefore \; \lambda = 24.1453}
\end{align*}

Thus, all the given options are incorrect.
\end{frame}

\end{document}