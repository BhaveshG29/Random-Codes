\let\negmedspace\undefined
\let\negthickspace\undefined
\documentclass[journal]{IEEEtran}
\usepackage[a5paper, margin=10mm, onecolumn]{geometry}
%\usepackage{lmodern} 
\usepackage{tfrupee} 

\setlength{\headheight}{1cm} 
\setlength{\headsep}{0mm}     

\usepackage{gvv-book}
\usepackage{gvv}
\usepackage{cite}
\usepackage{amsmath,amssymb,amsfonts,amsthm}
\usepackage{algorithmic}
\usepackage{graphicx}
\usepackage{textcomp}
\usepackage{xcolor}
\usepackage{txfonts}
\usepackage{listings}
\usepackage{enumitem}
\usepackage{mathtools}
\usepackage{gensymb}
\usepackage{comment}
\usepackage[breaklinks=true]{hyperref}
\usepackage{tkz-euclide} 
\usepackage{listings}                                        
\def\inputGnumericTable{}                                 
\usepackage[latin1]{inputenc}                                
\usepackage{color}                                            
\usepackage{array}                                            
\usepackage{longtable}                                       
\usepackage{calc}                                             
\usepackage{multirow}                                         
\usepackage{hhline}                                           
\usepackage{ifthen}                                           
\usepackage{lscape}

\begin{document}

\bibliographystyle{IEEEtran}
\vspace{3cm}

\title{12.289}
\author{AI25BTECH11003 - Bhavesh Gaikwad}
{\let\newpage\relax\maketitle}

\renewcommand{\thefigure}{\theenumi}
\renewcommand{\thetable}{\theenumi}
\setlength{\intextsep}{10pt} 

\numberwithin{equation}{enumi}
\numberwithin{figure}{enumi}
\renewcommand{\thetable}{\theenumi}


\textbf{Question}: The approximate eigenvalue of the matrix 
$$\vec{A} = \myvec{15 & 4 & 3 \\ 10 & 12 & 6 \\ 20 & 4 & 2}$$
obtained after two iterations of Power method, with the initial vector $\myvec{1 & 1 & 1}^\top$ is 

\hfill{(MA 2012)}

(a) 7.768 $\qquad$ (b)9.468 $\qquad$ (c)10.548 $\qquad$ (d)19.468
\bigskip
 
\textbf{Solution:}\\
Given: 
$$\vec{A} = \myvec{15 & 4 & 3 \\ 10 & 12 & 6 \\ 20 & 4 & 2}$$
$$\text{Let the initial vector be} \; \vec{x}_0 = \myvec{1 \\ 1 \\ 1}$$

\begin{equation}
\vec{x}_{1} = \vec{A}\vec{x}_0 = \myvec{22 \\ 28 \\ 26}    
\end{equation}


Let Normalization vector: $\vec{y}_i = \dfrac{1}{x_{max}}\vec{x}_i$\\
where, $x_{max}$ is the largest element in $\vec{x}_i$

\begin{equation}
\vec{y}_1 = \frac{1}{28}\vec{x}_1   
\end{equation}

\begin{equation}
\vec{x}_2 = \vec{A}\vec{y}_1 = \frac{1}{7}\myvec{260 \\ 356 \\ 302} 
\end{equation}

\begin{equation}
    \vec{y}_2 = \frac{1}{356} \vec{x}_2 = \frac{1}{1246}\myvec{130 \\ 178 \\ 151}
\end{equation}

\newpage

Let $\lambda$ be the Dominant eigenvalue.\\

By Rayleigh-Quotient,
\begin{equation}
    \lambda = \dfrac{\vec{y}_2^\top\vec{A}\vec{y}_2}{\vec{y}_2^\top\vec{y}_2}
\end{equation}

\begin{align*}
    \boxed{\lambda = 24.1453}
\end{align*}

Thus, all the given options are incorrect.
\end{document}  
