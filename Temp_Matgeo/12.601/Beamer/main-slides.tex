\documentclass{beamer}
\usepackage[utf8]{inputenc}

\usetheme{Madrid}
\usecolortheme{default}
\usepackage{amsmath,amssymb,amsfonts,amsthm}
\usepackage{txfonts}
\usepackage{tkz-euclide}
\usepackage{listings}
\usepackage{adjustbox}
\usepackage{array}
\usepackage{tabularx}
\usepackage{gvv}
\usepackage{lmodern}
\usepackage{circuitikz}
\usepackage{tikz}
\usepackage{graphicx}

\setbeamertemplate{page number in head/foot}[totalframenumber]

\usepackage{tcolorbox}
\tcbuselibrary{minted,breakable,xparse,skins}



\definecolor{bg}{gray}{0.95}
\DeclareTCBListing{mintedbox}{O{}m!O{}}{%
  breakable=true,
  listing engine=minted,
  listing only,
  minted language=#2,
  minted style=default,
  minted options={%
    linenos,
    gobble=0,
    breaklines=true,
    breakafter=,,
    fontsize=\small,
    numbersep=8pt,
    #1},
  boxsep=0pt,
  left skip=0pt,
  right skip=0pt,
  left=25pt,
  right=0pt,
  top=3pt,
  bottom=3pt,
  arc=5pt,
  leftrule=0pt,
  rightrule=0pt,
  bottomrule=2pt,
  toprule=2pt,
  colback=bg,
  colframe=orange!70,
  enhanced,
  overlay={%
    \begin{tcbclipinterior}
    \fill[orange!20!white] (frame.south west) rectangle ([xshift=20pt]frame.north west);
    \end{tcbclipinterior}},
  #3,
}
\lstset{
    language=C,
    basicstyle=\ttfamily\small,
    keywordstyle=\color{blue},
    stringstyle=\color{orange},
    commentstyle=\color{green!60!black},
    numbers=left,
    numberstyle=\tiny\color{gray},
    breaklines=true,
    showstringspaces=false,
}
%------------------------------------------------------------

\title
{12.601}
\date{September 28, 2025}
\author 
{AI25BTECH11003 - Bhavesh Gaikwad}



\begin{document}


\frame{\titlepage}
\begin{frame}{Question}
The matrix $\myvec{1 & 1 & 2 \\ 0 & 1 & 0 \\ 1 & 2 & 1}$, one of the eigen values is 1. The eigen vectors corresponding to the eigne value 1 are:
\hfill{(CS 2016)}\\

\begin{itemize}
    \item[a)] $\alpha\myvec{4 & -2 & 1}, \, \alpha \neq 0, \; \alpha \epsilon \mathbb{R}$
    \item[b)] $\alpha\myvec{-4 & 2 & 1}, \, \alpha \neq 0, \; \alpha \epsilon \mathbb{R}$
    \item[c)]$\alpha\myvec{-2 & 0 & 1}, \, \alpha \neq 0, \; \alpha \epsilon \mathbb{R}$
    \item[d)]$\alpha\myvec{2 & 0 & 1}, \, \alpha \neq 0, \; \alpha \epsilon \mathbb{R}$
\end{itemize}
\end{frame}


\begin{frame}[fragile]
    \frametitle{Theoretical Solution}
Given: $\lambda = 1$, $Let \; \vec{A} = \myvec{1 & 1 & 2 \\ 0 & 1 & 0 \\ 1 & 2 & 1}$

Let $\vec{v}$ be the corresponding eigenvector.

\begin{align}
    \Rightarrow \; \vec{A}\vec{v} &= (1)\vec{v} \\ 
    (\vec{A} - \vec{I})\vec{v} &= \myvec{0 & 0 & 0}
\end{align}

\begin{equation}
\myvec{0 & 1 & 2 \\ 0 & 0 & 0 \\ 1 & 2 & 0}\vec{v} = \myvec{0 & 0 & 0}    
\end{equation}

Let $\vec{v} = \myvec{v_1 & v_2 & v_3}$\\
\end{frame}

\begin{frame}[fragile]
\frametitle{Theoretical Solution}
    Substituting value of $\vec{v}$ in Equation 3,
\begin{equation}
   \myvec{0 & 1 & 2 \\ 0 & 0 & 0 \\ 1 & 2 & 0}\myvec{v_1 & v_2 & v_3} = \myvec{0 & 0 & 0} 
\end{equation}

\begin{align}
    Row-1 \rightarrow \; v_2 + 2v_3 &= 0\\
    Row-2 \rightarrow \; 0 + 0 + 0&= 0 \text{ (Always true)}\\
    Row-3 \rightarrow \; v_1 + 2v_2 &= 0
\end{align}

\newpage

Let $v_3 = \alpha$ (Free parameter)\\
Substituting value of $v_3$ in Equations 5 and 7
\begin{equation}
\therefore \; v_2 = -2\alpha \, \&  \, v_1 = 4\alpha  
\end{equation}
\begin{equation}
    \therefore \; \vec{v} = \alpha \myvec{4 & -2 & 1} 
\end{equation}
\begin{align*}
    \boxed{\text{Thus, Option-A is correct. }}
\end{align*}
\end{frame}
\end{document}