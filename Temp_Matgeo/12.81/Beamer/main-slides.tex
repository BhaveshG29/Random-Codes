\documentclass{beamer}
\usepackage[utf8]{inputenc}

\usetheme{Madrid}
\usecolortheme{default}
\usepackage{amsmath,amssymb,amsfonts,amsthm}
\usepackage{txfonts}
\usepackage{tkz-euclide}
\usepackage{listings}
\usepackage{adjustbox}
\usepackage{array}
\usepackage{tabularx}
\usepackage{gvv}
\usepackage{lmodern}
\usepackage{circuitikz}
\usepackage{tikz}
\usepackage{graphicx}

\setbeamertemplate{page number in head/foot}[totalframenumber]

\usepackage{tcolorbox}
\tcbuselibrary{minted,breakable,xparse,skins}



\definecolor{bg}{gray}{0.95}
\DeclareTCBListing{mintedbox}{O{}m!O{}}{%
  breakable=true,
  listing engine=minted,
  listing only,
  minted language=#2,
  minted style=default,
  minted options={%
    linenos,
    gobble=0,
    breaklines=true,
    breakafter=,,
    fontsize=\small,
    numbersep=8pt,
    #1},
  boxsep=0pt,
  left skip=0pt,
  right skip=0pt,
  left=25pt,
  right=0pt,
  top=3pt,
  bottom=3pt,
  arc=5pt,
  leftrule=0pt,
  rightrule=0pt,
  bottomrule=2pt,
  toprule=2pt,
  colback=bg,
  colframe=orange!70,
  enhanced,
  overlay={%
    \begin{tcbclipinterior}
    \fill[orange!20!white] (frame.south west) rectangle ([xshift=20pt]frame.north west);
    \end{tcbclipinterior}},
  #3,
}
\lstset{
    language=C,
    basicstyle=\ttfamily\small,
    keywordstyle=\color{blue},
    stringstyle=\color{orange},
    commentstyle=\color{green!60!black},
    numbers=left,
    numberstyle=\tiny\color{gray},
    breaklines=true,
    showstringspaces=false,
}
%------------------------------------------------------------

\title
{12.81}
\date{September 27, 2025}
\author 
{AI25BTECH11003 - Bhavesh Gaikwad}



\begin{document}


\frame{\titlepage}
\begin{frame}{Question}
Let $\vec{M}$ be a $3\times3$ real symmetric matrix with eigenvalues -1, 1, 2 and the corresponding unit eigenvectors $\vec{u}, \vec{v},\vec{w}$, respectively. Let $\vec{x}$ and $\vec{y}$ be two vectors in $\mathbb{R}^3$ such that
$$\vec{MX} = \vec{u} + 2(\vec{v} + \vec{w}) \text{ and } \vec{M}^2\vec{y} = \vec{u} - (\vec{v} + 2\vec{w})$$
Considering the usual inner product in $\mathbb{R}^3$, the value of $|\vec{x} + \vec{y}|^2$, where $|\vec{x} + \vec{y}|$ is the
length of the vector $\vec{x} + \vec{y}$, is

\hfill{(ST 2022)}

a) 1.25 $\qquad$ b) 0.25 $\qquad$ c) 0.75 $\qquad$ d) 1
\end{frame}


\begin{frame}[fragile]
    \frametitle{Theoretical Solution}
\begin{equation}
\text{Let }\vec{P} = \brak{\vec{u} \, \vec{v} \, \vec{w}} \text{ be the $3 \times 3$ orthogonal matrix of unit eigenvectors.}
\end{equation}



\begin{equation}
\text{Let } \vec{D} = \myvec{-1 & 0 & 0 \\ 0 & 1 & 0 \\ 0 & 0 & 2} \text{ be the diagonal matrix of eigenvalues.}
\end{equation}


Since $\vec{M}$ is symmetric with these eigenvectors and eigenvalues:\\
Eigen-Decomposition:
\begin{equation}
\vec{M} = \vec{P}\vec{D}\vec{P}^\top
\end{equation}


Let
\begin{align}
\vec{x} &= \vec{P}\vec{\alpha}, \quad \vec{y} = \vec{P}\vec{\beta}
\end{align}

Given:
\begin{equation}
\vec{M}\vec{x} = \vec{u} + 2\brak{\vec{v} + \vec{w}}
\end{equation}
\end{frame}

\begin{frame}[fragile]
    \frametitle{Theoretical Solution}
\begin{equation}
\vec{u} + 2\brak{\vec{v} + \vec{w}} = \vec{P} \myvec{1 \\ 2 \\ 2}
\end{equation}

\begin{equation}
\vec{M}\vec{x} = \vec{P}\vec{D}\vec{P}^\top \vec{P}\vec{\alpha} = \vec{P}\vec{D}\vec{\alpha}
\end{equation}

\begin{align}
\vec{P}\vec{D}\vec{\alpha} &= \vec{P} \myvec{1 \\ 2 \\ 2} \implies \vec{D}\vec{\alpha} = \myvec{1 \\ 2 \\ 2} \\
&\implies \myvec{-1 & 0 & 0 \\ 0 & 1 & 0 \\ 0 & 0 & 2} \vec{\alpha}  = \myvec{1 \\ 2 \\ 2}
\end{align}

\begin{equation}
\vec{\alpha} = \myvec{-1 \\ 2 \\ 1}
\end{equation}
\end{frame}

\begin{frame}[fragile]
    \frametitle{Theoretical Solution}
\begin{equation}
\therefore \; \vec{x} = \vec{P} \myvec{-1 \\ 2 \\ 1}
\end{equation}


Given:
\begin{equation}
\vec{M}^2\vec{y} = \vec{u} - \brak{\vec{v} + 2\vec{w}} = \vec{P} \myvec{1 \\ -1 \\ -2}
\end{equation}

\begin{equation}
\vec{M}^2\vec{y} = \brak{\vec{P}\vec{D}\vec{P}^\top}\brak{\vec{P}\vec{D}\vec{P}^\top}\vec{P}\vec{\beta} = \vec{P}\vec{D}^2\vec{\beta}
\end{equation}
\end{frame}

\begin{frame}[fragile]
    \frametitle{Theoretical Solution}
\begin{align}
\vec{P}\vec{D}^2\vec{\beta} &= \vec{P} \myvec{1 \\ -1 \\ -2} \implies \vec{D}^2\vec{\beta} = \myvec{1 \\ -1 \\ -2} \\
&\implies \myvec{1 & 0 & 0 \\ 0 & 1 & 0 \\ 0 & 0 & 4} \vec{\beta} = \myvec{1 \\ -1 \\ -2}
\end{align}


\begin{equation}
\vec{\beta} = \myvec{1 \\ -1 \\ -\frac{1}{2}}
\end{equation}

\begin{equation}
\therefore \; \vec{y} = \vec{P} \myvec{1 \\ -1 \\ -\frac{1}{2}}
\end{equation}

\end{frame}

\begin{frame}[fragile]
    \frametitle{Theoretical Solution}
\begin{align}
\vec{x} + \vec{y} &= \vec{P}\vec{\alpha} + \vec{P}\vec{\beta} = \vec{P}\brak{\vec{\alpha} + \vec{\beta}}
\end{align}

Since $\vec{P}$ is an orthogonal matrix and $\vec{u}, \, \vec{v}, \, \vec{w}$ are unit eigenvectors,
\begin{align}
\norm{\vec{x} + \vec{y}}^2 &= \norm{\vec{P}(\vec{\alpha} + \vec{\beta})}^2 = \left( \vec{\alpha} + \vec{\beta}\right)^\top (\vec{\alpha} + \vec{\beta})\\
\norm{\vec{x} + \vec{y}}^2 &=  \vec{\alpha}^\top\vec{\alpha} + \vec{\alpha}^\top\vec{\beta} + \vec{\beta}^\top\vec{\alpha} + \vec{\beta}^\top\vec{\beta}\\
\norm{\vec{x} + \vec{y}}^2 &= \norm{\vec{\alpha}}^2 + 2\vec{\alpha}^\top\vec{\beta} +
\norm{\vec{\beta}}^2\\
\norm{\vec{x} + \vec{y}}^2 &= 6 - 7 + \dfrac{9}{4}
\end{align}

\begin{align*}
    \boxed{\norm{\vec{x} + \vec{y}}^2 = 1.25}
\end{align*}

\begin{equation}
\boxed{\text{Thus, Option-A is correct.}}
\end{equation}
\end{frame}

\end{document}